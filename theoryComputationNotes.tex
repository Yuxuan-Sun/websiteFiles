\documentclass[a4paper]{article}

\usepackage[utf8]{inputenc}
\usepackage{amssymb, amsfonts, latexsym, amsthm, amsmath, framed}
\usepackage{esvect}
\usepackage[parfill]{parskip}
\usepackage{amsmath, amssymb, framed, tcolorbox}
\tcbuselibrary{theorems}

\usepackage[hidelinks,colorlinks=true,linkcolor=blue,citecolor=blue]{hyperref}

\newcommand{\ba}{\backslash}
\newcommand{\Q}{\mathbb{Q}}
\newcommand{\C}{\mathbb{C}}
\newcommand{\R}{\mathbb{R}}
\newcommand{\N}{\mathbb{N}}
\newcommand{\Z}{\mathbb{Z}}
\newcommand{\F}{\mathbb{F}}
\newcommand{\rank}{\text{rank}}

% figure support
\usepackage{import}
\usepackage{xifthen}
\pdfminorversion=7
\usepackage{pdfpages}
\usepackage{transparent}
\newcommand{\incfig}[1]{%
	\def\svgwidth{\columnwidth}
	\import{./figures/}{#1.pdf_tex}
}
\newcounter{mytheorem}[section] \def\themytheorem{\thesection.\arabic{mytheorem}}

\definecolor{my-yellow}{cmyk}{0,0.2,0.7,0,1.00}
\definecolor{my-blue}{cmyk}{0.80, 0.13, 0.14, 0.04, 1.00}
\definecolor{my-green}{cmyk}{0.4,0,0.4,0,1.00}
\tcbset{
defstyle/.style={fonttitle=\bfseries\upshape, colback=my-yellow!5,colframe=my-yellow!80!black},
theostyle/.style={fonttitle=\bfseries\upshape, colback=my-blue!5,colframe=my-blue!80!black},
corstyle/.style={fonttitle=\bfseries\upshape, colback=my-green!5,colframe=my-green!80!black},
}


\tcbmaketheorem{Definition}{Definition}{defstyle}{mytheorem}{def}
\tcbmaketheorem{Theorem}{Theorem}{theostyle}{mytheorem}{theo}
\tcbmaketheorem{Corollary}{Corollary}{corstyle}{mytheorem}{cor}

\usepackage{tcolorbox}
\usepackage{tikz}
\tcbuselibrary{theorems, skins, breakable}

\pdfsuppresswarningpagegroup=1

\title{Theorey of Computation Notes}
\author{Yuxuan Sun}
\date{Spring 2022}


\begin{document}

\maketitle

\tableofcontents

\newpage

\section{regular languages}

\begin{Definition}{regular language}
	A language is called a regular language if some finite automaton recognizes it.
\end{Definition}

\section{Non-regular languages}

\begin{Theorem}{Pumping Lemma}{pumpingLemma}
If $L$ is a regular language, then there is a positive integer $n$ (typically, $n$ is the number of states of the DFA accepting $L$ ) such that, if $x \in L$ and $|x| \geq n$, then there exist $u, v, w \in \Sigma^{*}$ such that $x=u v w$ and:
\begin{enumerate}
\item $\left| uv \right|\le n $ 
\item $\left| u \right| > 0 $ 
\item for each integer $m \ge 0$, $uv^mw\in L$
\end{enumerate}
\end{Theorem}

\begin{Corollary}{infinite}{pumpingInfinite}
Let the regular language $L$ be accepted by a DFA with  $n$ states. Then  $L$ is infinite if and only if there is  $x \in L$ s.t. $n \le \left| x \right| < 2n $.
\end{Corollary}

\begin{Theorem}{Complement}{pumpingComplement}
The class of regular languages is closed under complement.
	
\end{Theorem}

\begin{Definition}{Indistinguishable Strings}{indisting}
Let $L$ a language over $\Sigma$ and let $x, y \in \Sigma^{*}$. We say that $x$ and $y$ are indistinguishable with respect to $L$ and we write $x \approx_{L} y$ if, for all $z \in \Sigma^{*}$, either both $x z$ and $y z \in L$ or neither is. Furthermore, $\approx_{L}$ can be proved to be an equivalence relation on $\Sigma^{*}$.
\end{Definition}

\begin{Theorem}{Myhill-Nerode}{myhillNerode}
A language $L$ is regular if and only if the number of equivalence classes of $\approx_{L}$ is finite.
	
\end{Theorem}

\newpage
\subsection{Exercises using pumping lemma}

The followings are proofs of $L$ NOT being a regular language using the pumping lemma.

\begin{tcolorbox}[enhanced,breakable,colback=white]
$L = \{ x \in (a+b)^* | x \neq x^R \}$
\end{tcolorbox}
Prove by contradiction.

Suppose $L$ is regular. Then there exists $n \in  \N$ s.t. let $x = a^n b a^{n+n!}$. Because $n\neq n+n!$, $x \in L$. Because  $|x| = n+1+n+n! > n$, there exists  $u,v,w \in  \Sigma^*$ s.t. $x = uvw$ 

Based on the pumping lemma, we know that  $|uv|\le n$, based on the construction of $x$,  $v$ could only contain  $a$'s, denote as $v=a^i$ for some  $i \in \Z_{>0}$. Because $i$ is an interger that's smaller than  $n$, we know that  $n!/i$ will be an integer, denoted as $m$. i.e.  $im=n!$. For later reference, denote  $u=a^j,w=a^kba^{n+n!}$, where  $j,k \in \Z_{\ge 0}$, and $j+i+k=n$

Notice that $uv^{m+1}w=uvv^mw=a^{j+i+mi+k}ba^{n+n!}$, since we know $j+i+k=n$ and  $mi=n!$, we have  $uv^{m+1}w=a^{n+n!}ba^{n+n!}$, which is not in $L$. We have a contradiction to the pumping lemma, thus  $L$ is not regular.

\bigskip

\begin{tcolorbox}[enhanced,breakable,colback=white]
$L=\left\{a^{i} b^{j} \mid \frac{1}{2}(j+1) \leq i \leq \frac{1}{2}(3 j-1), i, j \in \mathbb{Z}, j \geq 0\right\}$
\end{tcolorbox}
The language could be rewritten as $L=\left\{a^{i} b^{j} \mid (j+1) \leq 2i \leq (3 j-1), i, j \in \mathbb{Z}, j \geq 0\right\}$ so that it's easier to check whether our string is in the language. 

Prove by contradiction.

Suppose $L$ is regualr. Then there exists $n\in \N$ s.t. let $x=a^nb^n$, the inequality holds as long as  $n\ge 1$ so $x \in L$. Beacuse $|x|=2n>n$, there exists  $u,v,w \in \Sigma^*$ s.t. $x=uvw$

Based on the pumping lemma, we know that $|uv|\le n$, based on the construction of $x$,  $u,v$ could only contain  $a$'s. Denote  $u=a^i,v=a^j$ where  $i,j \in \Z_{\ge 0}$ but $j\neq 0$. Denote $w=a^kb^n$,  $k \in \Z_{\ge 0}$.

Find any $m$ s.t.  $mj > 3n$, since  $j\neq 0$ we know we could find such $m$. Then notice that  $uv^mw=a^{i+mj+k}b^n$. It is not in $L$ since  $2(i+mj+k) > 3n > 3n-1$, which violates the inequality of language. We have a contradiction to the pumping lemma, thus  $L$ is not regular.

\bigskip

\begin{tcolorbox}[enhanced,breakable,colback=white]
$L = \left\{x \in(a+b)^{*} \mid n_{a}(x) \neq 10 n_{b}(x)\right\}$
\end{tcolorbox}
We need to prove by its complement, namely \[
	\overline{L} = \left\{ x \in (a+b)^* | n_a(x) = 10n_b(x) \right\} 
\]
Prove by contradiction.

Suppose $\overline{L}$ is regular. Then there exists $n \in \N$ s.t. $x = a^{10n}b^n$. Because $10n=10n$,  $x \in \overline{L}$. Because $|x|=10n+n>n$, there exists  $u,v,w \in \Sigma^*$ s.t. $x=uvw$.

Based on the pumping lemma, we know that  $|uv|\le n$, based on the construction of $x$,  $u,v$ could only contain  $a$'s. Denote  $u=a^i,v=a^j,w=a^kb^n$ where  $i,j,k \in \Z_{\ge 0}$ but  $j\neq 0$, and $i+j+k=10n$.

Take any  $m\ge 1$, then $uv^mw=a^{i+mj+k}b^n$. Because $j\neq 0$ and $m \neq 0$, we know that $i+mj+k>i+j+k$, i.e. $i+mj+k>10n$, so the inequality is broken as  $i+mj+k \neq 10n$. We have a contradiction to the pumping lemma, thus $\overline{L}$ is not regular, and according to theorem in class, $L$ is not regular.

\bigskip

\begin{tcolorbox}[enhanced,breakable,colback=white]
$L=\left\{a^{i} b^{j} c^{k}\mid j= |i-k|, i, j, k \in \mathbb{Z}, i, k \geq 0\right\}$
\end{tcolorbox}
Prove by contradiction.

Suppose $L$ is regular. Then there exists  $n \in  \N$ s.t. $x = a^{2n}b^nc^n$. Because  $n=|2n-n|$,  $x \in L$. Because $|n|=2n+n+n=4n>n$, there exists  $u,v,w \in \Sigma^*$ s.t. $x=uvw$.

Based on the pumping lemma, we know that  $|uv|\le n$, based on the construction of $x$,  $u,v$ could only contain  $a$'s. Denote  $u=a^i,v=a^j,w=a^kb^nc^n$ where  $i,j,k \in \Z_{\ge 0}$ but $j\neq 0$, and $i+j+k=2n$.

Take any  $m\ge 1$, then $uv^mw=a^{i+mj+k}b^nc^n$. Because  $m \neq 0$ and $j\neq 0$, $i+mj+k > 2n$. In other words,  $|i+mj+k-n|>|2n-n|$, so  $|i+mj+k-n|\neq n$. $uv^mw \not\in L$. We have a contradiction to the pumping lemma, thus $L$ is not regular. 

\bigskip

\begin{tcolorbox}[enhanced,breakable,colback=white]
$L = \left\{a^{i} b^{j} c^{k} \mid i, j, k \in \mathbb{Z}, i, j, k \geq 0 \text { such that if } i=1, \text { then } j=k\right\}$ 
\end{tcolorbox}
We need to prove by its complement, namely \[
	\overline{L} = \left\{ a^ib^jc^k \mid i,j,k\in \Z, i,j,k\ge 0 \text{ s.t. } i=1 \text{ and } j \neq k \right\} 
\] 

Prove by contradiction.

Suppose $\overline{L}$ is regular. Then there exists $n \in \N$ s.t. $x=ab^nc^{n+n!}$. Becuase  $n\neq n+n!$ and $i=1$,  $x \in \overline{	L}$. Because $|x|=1+n+n+n\mapsto n$, there exists $u,v,w \in \Sigma^*$ s.t. $x=uvw$.

Based on the pumping lemma, we know that  $|uv|\le n$, based on the construction of $x$, there are several possible situations of $v$, specifically:
\begin{enumerate}
\item $v=a$

	This will give us a contradiction because  $uv^2w = a^2w$, but in  $\overline{L}$ we don't allow $i$(the power of  $a$) to be anything but  $1$.
\item  $v=ab^i$ for some  $i \in \Z_{>0}$ 

	This will give us a contradiction because $uv^2w=ab^iab^iw$, which is clearly not in  $\overline{L}$ because the language doesn't allow sth like $aba$.
\item  $v=b^j$ for some  $j \in \Z_{>0}$

	The contradiction is similar with the one in  \textbf{1.4.1}. Denote $u=ab^i,w=b^kc^{n+n!}$ where $i,k \in \Z_{\ge 0}$, and $i+j+k=n$. Since  $j<n$,  $n!/j$ will give us an integar, let's denote it as  $m$, i.e.  $mj=n!$.

	Notice  $uv^{m+1}w = ab^{i+j+k+mj}c^{n+n!}$, based on what we know,  $i+j+k+mj=n+n!$, i.e. $uv^{m+1}w \not\in \overline{L}$. We have a contradiction to the pumping lemma, based on the theorem in class, we know $\overline{L}$ not regular implies $L$ is not regular either.
	
\end{enumerate}

That all possible situations for $v$ because  $|uv|\le n$ so there is no way $v$ could contain  $c$.


\newpage
\section{Context-Free Grammars}
\begin{Definition}{Context-Free Grammars}{CFG}
	A \textbf{context-free} grammar is a 4-tuple $G = \left( V, \Sigma, S, P \right) $s.t.:
\begin{enumerate}
\item $V$ is a finite set of variables, $S \in V$ is the start variable
\item $\Sigma$ is a finite set of terminal symbols or teminals s.t.  $V \cap \Sigma  = \varnothing$
\item $P$ is a finite seet, whose elements are  \textbf{grammar rules} or \textbf{productions} in the form \[
A \to \alpha
\] 
where $A \in  V$ and $\alpha \in (V \cup \Sigma)^*$
	
\end{enumerate}
	
\end{Definition}

\begin{Definition}{The Language Generated by a CFG}{CFL}
	If $G=\left( V, \Sigma, S,P \right) $ is a CFG, the language generated by  $G$ is  \[
		L(G) = \left\{ x \in \Sigma^* \mid S \Longrightarrow_G^* x \right\} 
	\] 
	Language $L$ is a context-free language if there is a CFG  $G$ s.t.  $L = L(G)$
\end{Definition}


\begin{Definition}{derivation tree/parse tree}{parseTree}
Let $G$ be a CFG. The  \textbf{derivation tree} for $G$ is an ordered tree s.t.
 \begin{enumerate}
\item the root is labeled $S$
\item every leaf has a label from  $\Sigma \cup \left\{ \epsilon \right\} $ 
\item every interior vertex has a label from $V$
\item if a vertex has label  $A$, and its children are labeled (left to right ) $a_1,a_2,\ldots,a_n$, where $a_j \in V\cup \Sigma \cup \left\{ \epsilon \right\} $ for $j = 1,2,3,\ldots,n$, then $P$ contains a production of the form  $P\to a_1a_2\ldots a_n$
\end{enumerate}
\end{Definition}
\begin{Definition}{yield of a tree}{yield}
the string of terminals obtained by reading the leaves of the tree from left to right, omitting any $\epsilon$.
	
\end{Definition}

\begin{Theorem}{derivation tree and yield}{yield}
	If  G  is a CFG, then, for every $x \in L(G)$, there exists a derivation tree of  $G$ whose yield is $x$ . Conversely, the yield of any derivation tree is in  $L(G)$
\end{Theorem}

	\begin{Definition}{leftmost derivation}{LMD}
A derivation in a CFG is a leftmost derivation if, at each step, a production is applied to the leftmost variable-occurrence in the current string.
\end{Definition}

\begin{Theorem}{equivalent statement of $x \in L(G)$}{equivL}
\begin{enumerate}
	\item $x$ has more than one derivation tree
	\item  $x$ has more than one leftmost derivation
	\item  $x$ has more than one rightmost derivation
\end{enumerate}
\end{Theorem}

\begin{Definition}{ambiguous CFG}{ambiguousCFG}
	A CFG $G$ is ambiguous if, there exists  $x \in L(G)$ s.t. $x$ has more than one derivation tree.
\end{Definition}

\newpage
\section{Push-Down Autonoma}

\begin{Definition}{PDA}{}
$M=\left(Q, \Sigma, \Gamma, \delta, q_{0}, Z, F\right)$ where
\begin{itemize}
	\item $Q$ is a finite set of states
	\item $\Sigma$ is a finite set which is called the input alphabet
	\item $\Gamma$ is a finite set which is called the stack alphabet
\item $\delta$ is a finite subset of $Q \times(\Sigma \cup\{\varepsilon\}) \times \Gamma \times Q \times \Gamma^{*}$, the transition relation
\item $q_{0} \in Q$ is the start state
\item $Z \in \Gamma$ is the initial stack symbol
\item  $F \subseteq Q$ is the set of accepting states
\end{itemize}
\end{Definition}

\begin{Definition}{Configurations and Moves}{}
	Let $M=\left(Q, \Sigma, \Gamma, q_{0}, Z_{0}, F, \delta\right)$ be a PDA

	A  \textbf{configuration/instananeous discription} of $M$ is a triplet  $(p, w, \gamma)$ s.t. when  $M$ is on state  $p \in Q$, the part of the input string that is about to be read is string $w \in \Sigma^*$ and, the contents of the whole stack are given my string $\gamma \in \Gamma^*$ 

	Two configurations $(p, \sigma w, Z \alpha)$ and $(q, w, \gamma \alpha)$ (for $\left.p, q \in Q, \sigma \in \Sigma_{\varepsilon}, w \in \Sigma^{*}, Z \in \Gamma_{\varepsilon}, \alpha, \gamma \in \Gamma^{*}\right)$ are said to form a move in one step, written as \[
	(p, \sigma w, Z \alpha) \vdash(q, w, \gamma \alpha)
	\] whenenver $\delta(p, \sigma, Z) \ni(q, \gamma)$
\end{Definition}

\begin{Definition}{chains of moves}{}
	Let $C_0, C_1, \ldots, C_n$ be a sequence of configurations, \textbf{a chain of moves in $n$ steps}, $C_{0} \vdash C_{1} \vdash \cdots \vdash C_{n}$, could be written as  \[
	C_{0} \vdash^{n} C_{n}
	\]  
\end{Definition}

\begin{Definition}{Right-, Left-, Regular and Linear Grammars}{}
	Let $G = \left( V, \Sigma, S, P \right) $ a CFG.

	$G$ is called  \textbf{left-linear} if all productions are of one of the two forms,
	\begin{align*}
		A & \to  xB \\
		A &\to x
	\end{align*} where $A, B \in V$ and $x \in \Sigma^*$

	(similar for  \textbf{left-linear})

	$G$ is called regular grammar if ti's either right- or left-linear

	 $G$ is called linear grammar if at most one variable can occur on the right side of any production, independently of its position.

\end{Definition}

\begin{Definition}{Chomsky Normal Form}{}
	A CFG $G = \left( V, \Sigma, S, P \right) $ is in \textbf{Chomsky normal form} if all productions are of the form \[
	A \to  BC
	\]  or \[
	A\to a
\] where $A,B,C \in V$ and $a \in \Sigma$ (notice $a \neq \epsilon$)
\end{Definition}

\begin{Theorem}{}{}
	Modify $G$'s productions, an equivalent CFG  $\hat{G}$ in Chomsky normal form can be created.
\end{Theorem}

\begin{Theorem}{}{}
	For every CFG $G$, there is a PDA  $M$ s.t.  $L(M) = L(G)$
	
\end{Theorem}

\begin{Theorem}{Pumping Lemma for Context-Free Languages}{}
	If $L$ is a context-free language over alphabet  $\Sigma$, then there is a positive integer  $n$ s.t., for every  $x \in L$ with $|x| \ge n$, $x$ can be written as  $x = uvwxy$ for some string  $u,v,w,x,y \in \Sigma^*$ satisfying:

	$|vwx| \le n$

	$|vx| \ge  1$, i.e. $v \neq  \epsilon$ or $x \neq  \epsilon$

	for every integer $m \ge  0$, $uv^m wx^m y \in L$
\end{Theorem}

\begin{Corollary}{}{}
	Let $L$ be a CFL and  $n$ the positive integer from the pumping lemma, then:

	$L \neq \varnothing$ iff there exsits $w \in L$ with $|w| < n$

	 $L$ is infinite iff there exists  $z \in L$ s.t. $n \le |z| < 2n$

\end{Corollary}


\begin{Theorem}{Ogden's Lemma}{}
	If $L$ is a context-free language over alphabet  $\Sigma$, then there is a opstivie integer  $n$ s.t. for every  $x \in L$ with $|x|\ge n$, if we mark at least $n$ symbols of  $x$,  $x$ can be written as  $x = uvwxy$, for some strings  $u,v,w,x,y \in \Sigma*$ satisfying

the string $vwx$ contains at most  $n$ marked symbols

the string  $vx$ contains at least one marked symbol

for every integer  $m \ge 0$, $uv^m w x^m y \in L$
\end{Theorem}

\begin{Theorem}{}{}
	The class of CFLs is closed under union, concatenation, and Kline star

	the class of CFLs is not closed under intersection and complementation

the intersection of a CFL with a regular language is a CFL
	
\end{Theorem}

\newpage
\section{decidability}

\begin{Definition}{Turing-decidable, decidable}{}
	A language is \textbf{Turing-decidable} or \textbf{decidable} if some Turing machine decides it: always make a decision to accept of reject.   
\end{Definition}

\begin{Theorem}{$A_{DFA}$ is a decidable language}{}
	Let \[
	A_{DFA} = \left\{ \langle B, w \rangle  \mid  B \text{ is a DFA that accepts input string } w \right\}.
	\] 
\end{Theorem}
\textbf{Proof Idea}

We need to present a TM $M$ that decides  $A_{DFA}$.

Let $M$ be a turning machine such that:

On input  $\langle B, w \rangle $, where $B$ is a DFA and  $w$ is a string:
 \begin{enumerate}
	 \item simulate $B$ on input  $w$
	\item if the simulation ends in an accept state, \it{accepts}. If it ends in a non-accepting state, \it{reject}.
\end{enumerate}


\begin{Theorem}{$A_{NFA}$ is a decidable language}{}
	\[
		A_{NFA} := \left\{ \langle B,w \rangle \mid B \text{ is an NFA that accepts input string } w  \right\} 
	\] 
\end{Theorem}

\begin{Theorem}{$A_{REX}$ is a decidable language}{}
	\[ A_{REX} := \left\{ \langle R,w \rangle \mid R \text{ is a regular expression that generates string } w  \right\} 
	\] 
\end{Theorem}

\begin{Theorem}{$E_{DFA}$ is a decidable language}{E_DFA} 
	\[
	E_{DFA} := \left\{ \langle A \rangle \mid A \text{ is a DFA and }L(A) = \varnothing  \right\} 
\] 
In other words, $A$ doesn't accept anything (not even the empty string).
\end{Theorem}

\textbf{Proof}

Design a turning machine $T$ s.t. given input  $\langle A \rangle $ where $A$ is a DFA:
 \begin{enumerate}
	 \item mark the start state of $A$
	\item repeeat until no new states get marked
	\item mark any state that has a transition coming into it from any state that is already market
	\item if no accept state is marked, accept; otherwise reject.
\end{enumerate}

\begin{Theorem}{$EQ_{DFA}$ is a decidable language.}{}
	\[
		EQ_{DFA} := \left\{ \langle A,B \rangle \mid A,B \text{ are DFAs and } L(A) = L(B)  \right\} 
	\] 
\end{Theorem}

\textbf{Proof}

Let's use Theorem \ref{theo:E_DFA}. To use it, we need to construct a DFA, denoted as $C$, s.t. $L(C) = \varnothing$.

To do so, we use \textbf{symmetric difference} of $L(A)$ and  $L(B)$, namely  \[
	L(C) = \left( L(A) \cap \overline{L(B)} \right) \cup \left( \overline{L(A)} \cap L(B) \right)  
\]  

Notice that if $L(A) \subseteq L(B)$, then $L(A) \cap \overline{L(B)} = \varnothing$, similar for the second part. Thus, if $L(A) = L(B) \iff L(C) = \varnothing$. Now we have our $L(C)$, the turning machine construction is easy.

Construct a turning machine  $F$ s.t. given input  $\langle A,B \rangle $, where $A,B$ are DFAs
 \begin{enumerate}
	 \item construct DFA $C$ as shown above
	 \item run a turning machine $T$ on  $C$ as in Theorem \ref{theo:DFA} on input  $\langle C \rangle $.
	\item if $T$ accepts, accept. If  $T$ rejects, reject.
\end{enumerate}

\begin{Theorem}{$A_{CFG}$ is a decidable language}{}
	\[
		A_{CFG} := \left\{ \langle G,w \rangle \mid G \text{ is a CFG that generates string } w \right\} 
	\] 
\end{Theorem}

\begin{Theorem}{$E_{CFG}$ is a decidable language}{}
	\[
		E_{CFG} := \left\{ \langle G \rangle \mid G \text{ is a CFG and } L(G) = \varnothing \right\} 
	\] 
\end{Theorem}

\subsection{undecidability}

\begin{Theorem}{$A_{TM}$ is undecidable}{}
	\[
		A_{TM} := \left\{ \langle M,w \rangle \mid M \text{ is a TM and } M \text{ accepts } w  \right\} 
	\] 
	Notice that $A_{TM}$ is turning-recognizable, i.e. we could build a turning machine that might loop when $M$ loops on  $w$. This tells us that turing-recognizable is more powerful than  decidable.
\end{Theorem}

\begin{Definition}{co-Turing-recognizable}{}
	A language is co-Turing-recognizable if it is the complement of a Turing-reconizable language. (its complement is Turing-recognizable).
\end{Definition}

\begin{Theorem}{}{}
	A language is decidable iff it is Turing-recognizable and co-Turing-recognizable.
\end{Theorem}
	in other words, A langugae is decidable iff it is Turing-recognizable and its complement is Turing-recognizable.

	in terms of undecidability, we just take the negation of the statement, namely: a language is undecidable iff it is not Turing-recognizable or its complement is not Turing-recognizable.

	\begin{Corollary}{$\overline{A_{TM}}$ is not Turing-recognizable}{}
	Prove by contradiction. If $\overline{A_{TM}}$ is Turing-recognizable, then by the theorem above, $A_{TM}$ should be decidable, yet we know it's not. Thus $\overline{A_{TM}}$ has to be NOT turing-recognizable.
\end{Corollary}

\begin{Theorem}{$E_{TM}$ is undecidable}{}
	\[
		E_{TM} := \left\{ \langle M \rangle \mid M \text{ is a TM and } L(M) = \varnothing  \right\} 
	\] 
\end{Theorem}

\begin{Theorem}{$REGULAR_{TM}$ is not decidable}{}
	\[
		REGULAR_{TM} = \left\{ \langle M \rangle \mid M \text{ is a TM and } L(M) \text{ is a regular language } \right\} 
	\] 
\end{Theorem}

\begin{Theorem}{$EQ_{TM}$ is not decidable}{}
	\[
		EQ_{TM} := \left\{ \langle M_1, M_2 \rangle \mid M_1, M_2 \text{ are TMs and } L(M_1) = L(M_2)  \right\} 
	\] 
\end{Theorem}

\newpage

\newpage
\section{reducibility}

\begin{Definition}{computable function}{}
	A function $f: \Sigma^* \to \Sigma^*$ is a \textbf{computable function} if some Turing machine $M$, on every input  $w$, halts with just  $f(w)$ on its tape. 
\end{Definition}

\begin{Definition}{mapping reducible}{}
	Language $A$ is  \textbf{mapping reducible} to language $B$, denoted as  $A \le_m B$, if there is a computable function $f: \Sigma^* \to \Sigma^*$, where for every $w$,  \[
	w \in A \Longleftrightarrow f(w) \in B .
	\]  The function $f$ is called the  \textbf{reduction} from $A$ to  $B$. 
\end{Definition}

\begin{Theorem}{}{}
	If $A \le_m B$ and $B$ is decidable, then  $A$ is decidable.
\end{Theorem}

Notice the contrapositive is:

\begin{Corollary}{}{}
	If $A  \le_m B$ and $A$ is undecidable, then  $B$ is undecidable.
\end{Corollary}

\begin{Theorem}{}{}
	If $A \le_m B$ and $B$ is turing-recognizable, then  $A$ is turing-recognizable.
\end{Theorem}

similarly, for its contrapostivie:

\begin{Corollary}{}{}
	If $A \le_m B$ and $A$ is not Turing-recognizable, then  $B$ is not Turing-recognizable.
\end{Corollary}

\newpage
\subsection{exercises}

\begin{tcolorbox}[enhanced,breakable,colback=white]
If $A\le_m B$ and $B$ is a regular language, does that imply that  $A$ is a regular language?
\end{tcolorbox}
\textbf{Answer: No}
Recall from corollary \textbf{5.23} from book that: if $A\le_m B$ and $A$ is undecidable, then  $B$ is undecidable.

Take the contra-positive of the corollary, we'll have: if  $B$ is decidable then either  $A \nleq_m B$ or $A$ is decidable.

Since we have  $B$, as a regular language, is decidable and  $A \le B$, we know that $A$ is decidable. Thus let's find any decidable language that's not a regular expression. Take \[
A = \left\{ a^n b^n \right\} 
\] we know it's decidable becuase it could be regonized by a PDA. To show $ A \le_m B$, we could have the following computable function: \[
f(w) = \begin{cases}
	0 & \text{if } w = a^n b^n \\
	1 & \text{otherwise}
\end{cases}
\] clearly that $ w \in A \Longleftrightarrow f(w) \in B$. Thus we've found a counter example.




\end{document}





















